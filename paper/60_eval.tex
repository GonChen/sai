\section{Evaluation} \label{evaluation}

To evaluate performance improvements of staging, we use the abstract interpreter
demonstrated in the paper to analyze some Scheme programs. We first implement
a desugaring transformation for a large subset of scheme that transforms into the
small language we used in the paper.
We use the same generic interface with a 0-CFA-like abstraction, and the unstage abstract
interpreter forms the baseline.

All of our evaluation benchmarks were performed on an Ubuntu 16.04 LTS (Linux kernel 4.4.0)
machine with 4 Intel Xeon Platinum 8168 CPU at 2.7GHz and 3 TiB of RAM.
Although the machine has 96 cores and 192 threads in total,
the abstract interpreters only use one thread to run all the benchmark programs.

Our evaluations show that the staged abstract interpreter performs well. In interesting
cases that are considered to be worst case scenarios for the control flow
analysis, the staged abstract interpeter outperforms the unstaged version by a wide
margin (up to 3x performance gain).

\subsection{Benchmarks}

Our evaluation is based on a suite of benchmark tests. Considering that our abstract
interpreters are implemented in Scala which will be affected by JVM warm up times, programs
tend to run faster after a number of runs. In order to minimize the influence of JVM adaptive optimizations,
we ran all experiments 10 times and take the statistical mean and median values of
the running times.

%on a suite of benchmark scheme programs that are also used in \cite{Johnson:2013:OAA:2500365.2500604, ashley:practical}. 
We used the following benchmark programs:
\begin{itemize}
    \item \textbf{fib:} a program that recursively calculates the $n$-th fibonacci number in
        exponential time.
    \item \textbf{church:} a program introduced by \todo{cite Dimitrios Vardoulakis and Olin Shivers. CFA2: a Context-Free approach
        to Control-Flow analysis. Logical Methods in Computer Science,
        7(2), 2011.} to test distributivity of multiplication for numbers in Church Encoding \todo{ cite }
    \item \textbf{kcfa3/kcfa-worst-case-n:} Benchmark programs that are supposed to
      be tough cases for CFA.
    \item \textbf{fermat:} Fermat and Solovay-Strassen primality testing in Scheme by
      Matthew Might.
    \item \textbf{rsa:} The RSA public key encryption algorithm in Scheme.
\end{itemize}

\subsection{Performance}

For each benchmark program, we ran both our unstaged and staged prototype abstract interpreters.
\todo{Figure show data.}

\begin{figure}[h]
\begin{tabular}{@{}llll@{}}
\toprule
    Program            & unstaged   & staged     & $\Delta$ of median \\ \midrule
    fib                & 1.020      & 2.046      & -50.14\%          \\
    rsa                & 35.928     & 145.306    & -75.27\%          \\
    kcfa3              & 2.934      & 5.671      & -48.27\%          \\
    church             & 884.800    & 545.736    & +62.13\%          \\
    fermat             & 15.673     & 68.790     & -77.22\%          \\
    kcfa-worst-16      & 320.689    & 301.564    & +6.34\%           \\
    kcfa-worst-32      & 3,685.615  & 1,854.461  & +98.74\%          \\
    kcfa-worst-64      & 49,422.666 & 14,849.725 & +232.82\%         \\
    \bottomrule
\end{tabular}
\caption{Evaluation result.} \label{evaluation_result}
\end{figure}

Figure~\ref{evaluation_result} shows the benchmark evaluation results. The "unstaged" and "staged" columns
show the median time to finish the abstract interpretation (in milliseconds). The "$\Delta_{median}$"
column shows the performance change from the unstaged abstract interpreter to the staged version.\\

However, there also exists some programs where the unstaged abstract interpreter yields 
faster results than the unstaged interpreter.
From the data shown above, we observed that 
However, in general, the unstaged abstract interpreter yields results faster
than the unstaged interpreter.

