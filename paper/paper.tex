%\documentclass[sigplan,nonacm]{acmart}\settopmatter{printfolios=true,printccs=false,printacmref=false}

%% For double-blind review submission, w/o CCS and ACM Reference (max submission space)
%\documentclass[acmsmall,review,anonymous]{acmart}\settopmatter{printfolios=true,printccs=false,printacmref=false}
\documentclass[acmsmall,review]{acmart}\settopmatter{printfolios=true,printccs=false,printacmref=false}
%% For double-blind review submission, w/ CCS and ACM Reference
%\documentclass[sigplan,review,anonymous]{acmart}\settopmatter{printfolios=true}
%% For single-blind review submission, w/o CCS and ACM Reference (max submission space)
%\documentclass[sigplan,review]{acmart}\settopmatter{printfolios=true,printccs=false,printacmref=false}
%% For single-blind review submission, w/ CCS and ACM Reference
%\documentclass[sigplan,review]{acmart}\settopmatter{printfolios=true}
%% For final camera-ready submission, w/ required CCS and ACM Reference
%\documentclass[acmsmall]{acmart}\settopmatter{}

%% Journal information
%% Supplied to authors by publisher for camera-ready submission;
%% use defaults for review submission.
\acmJournal{PACMPL}
\acmVolume{1}
\acmNumber{CONF} % CONF = POPL or ICFP or OOPSLA
\acmArticle{1}
\acmYear{2019}
\acmMonth{1}
\acmDOI{} % \acmDOI{10.1145/nnnnnnn.nnnnnnn}
\startPage{1}

%% Copyright information
%% Supplied to authors (based on authors' rights management selection;
%% see authors.acm.org) by publisher for camera-ready submission;
%% use 'none' for review submission.
\setcopyright{none}
%\setcopyright{acmcopyright}
%\setcopyright{acmlicensed}
%\setcopyright{rightsretained}
%\copyrightyear{2018}           %% If different from \acmYear

%% Bibliography style
\bibliographystyle{ACM-Reference-Format}
%% Citation style
\citestyle{acmauthoryear}  %% For author/year citations
%\citestyle{acmnumeric}     %% For numeric citations
%\setcitestyle{nosort}      %% With 'acmnumeric', to disable automatic
                            %% sorting of references within a single citation;
                            %% e.g., \cite{Smith99,Carpenter05,Baker12}
                            %% rendered as [14,5,2] rather than [2,5,14].
%\setcitesyle{nocompress}   %% With 'acmnumeric', to disable automatic
                            %% compression of sequential references within a
                            %% single citation;
                            %% e.g., \cite{Baker12,Baker14,Baker16}
                            %% rendered as [2,3,4] rather than [2-4].


%%%%%%%%%%%%%%%%%%%%%%%%%%%%%%%%%%%%%%%%%%%%%%%%%%%%%%%%%%%%%%%%%%%%%%
%% Note: Authors migrating a paper from traditional SIGPLAN
%% proceedings format to PACMPL format must update the
%% '\documentclass' and topmatter commands above; see
%% 'acmart-pacmpl-template.tex'.
%%%%%%%%%%%%%%%%%%%%%%%%%%%%%%%%%%%%%%%%%%%%%%%%%%%%%%%%%%%%%%%%%%%%%%


%% Some recommended packages.
\usepackage{booktabs}   %% For formal tables:
                        %% http://ctan.org/pkg/booktabs
\usepackage{subcaption} %% For complex figures with subfigures/subcaptions
                        %% http://ctan.org/pkg/subcaption

\usepackage[utf8]{inputenc}
\usepackage[T1]{fontenc}
\usepackage[scaled=0.83]{beramono}
\usepackage{amsmath}
\usepackage{amssymb}
%\usepackage{MnSymbol}
\usepackage{xcolor,colortbl}
\usepackage{url}
\usepackage{listings}
\usepackage{paralist}
%\usepackage[compact]{titlesec}
\usepackage[font={small}]{caption}
\usepackage{wrapfig}
\usepackage{enumitem}
\usepackage{multicol}
\usepackage{flushend}
\usepackage{bcprules}

\usepackage{tikz}
\usetikzlibrary{matrix}

\input{macros}

\newcommand{\updownarrows}{\mathbin\uparrow\hspace{-.5em}\downarrow}
\newcommand{\downuparrows}{\mathbin\downarrow\hspace{-.5em}\uparrow}

\begin{document}

%% Title information
%\title{Staged Abstract Interpreters}         %% [Short Title] is optional;
\title{Staged Abstract Interpreters}         %% [Short Title] is optional;
                                        %% when present, will be used in
                                        %% header instead of Full Title.
%\titlenote{with title note}             %% \titlenote is optional;
                                        %% can be repeated if necessary;
                                        %% contents suppressed with 'anonymous'
%\subtitle{Fast and Compositional Whole-Program Analysis for Free}                     %% \subtitle is optional
\subtitle{Fast and Modular Whole-Program Analysis via Meta-Programming}                     %% \subtitle is optional
%\subtitlenote{with subtitle note}       %% \subtitlenote is optional;
                                        %% can be repeated if necessary;
                                        %% contents suppressed with 'anonymous'

\iffalse
\author
[Guannan Wei, Yuxuan Chen, Tiark Rompf]
{
\vspace{-2ex}
Guannan Wei, Yuxuan Chen, Tiark Rompf\\
Purdue University
\vspace{-0.5ex}
}
\fi


%% Author information
%% Contents and number of authors suppressed with 'anonymous'.
%% Each author should be introduced by \author, followed by
%% \authornote (optional), \orcid (optional), \affiliation, and
%% \email.
%% An author may have multiple affiliations and/or emails; repeat the
%% appropriate command.
%% Many elements are not rendered, but should be provided for metadata
%% extraction tools.

%% Author with single affiliation.
\author{Guannan Wei}
\author{Yuxuan Chen}
\author{Tiark Rompf}
%\authornote{with author1 note}          %% \authornote is optional;
                                        %% can be repeated if necessary
%\orcid{nnnn-nnnn-nnnn-nnnn}             %% \orcid is optional
\affiliation{
  %\position{Position1}
  \department{Department of Computer Science}              %% \department is recommended
  \institution{Purdue University}            %% \institution is required
  %\streetaddress{305 N. University Street}
  %\city{West Lafayette}
  %\state{IN}
  %\postcode{47906}
  \country{USA}                    %% \country is recommended
}
\email{guannanwei@purdue.edu}          %% \email is recommended
\email{chen1797@purdue.edu}
\email{tiark@purdue.edu}          %% \email is recommended

%% Author with two affiliations and emails.
%\authornote{with author2 note}          %% \authornote is optional;
                                        %% can be repeated if necessary
%\authornote{with author1 note}          %% \authornote is optional;
                                        %% can be repeated if necessary
%\orcid{nnnn-nnnn-nnnn-nnnn}             %% \orcid is optional

\lstMakeShortInline[keywordstyle=,%
                    flexiblecolumns=false,%
                    %basewidth={0.56em, 0.52em},%
                    mathescape=false,%
                    basicstyle=\tt]@

%% Abstract
%% Note: \begin{abstract}...\end{abstract} environment must come
%% before \maketitle command
\begin{abstract}
  It is well known that a staged interpreter is a compiler: specializing the
interpreter to a given program produces an equivalent executable that runs faster.
This connection is known as the first Futamura projection.
It is even more widely known that an abstract interpreter is a program analyzer:
tweaking the interpreter to run on an abstract domain produces a sound static
analysis. What happens when we combine these two ideas, and apply staging to
an \emph{abstract} interpreter?

In this paper, we present a unifying framework that naturally extends the first
Futamura projection from concrete interpreters to abstract interpreters. Our
approach derives a sound staged abstract interpreter based on a
semantic-agnostic interpreter with type-level binding-time abstraction and
monadic abstraction. By using different instantiations of these abstractions,
the generic interpreter can flexibly behave in four modes: unstaged concrete
interpreter, staged concrete interpreter, unstaged abstract interpreter, or
staged abstract interpreter.
As an example of \emph{abstraction without regret}, the overhead of these
abstraction layers is eliminated in the generated code after staging.
We show that staging abstract interpreters is practical and useful to
optimize static analysis while requiring less engineering efforts and not
compromising soundness. We conduct three case studies, including a comparison
with \citeauthor{Boucher:1996:ACN:647473.727587}'s abstract compilation,
applications on various control-flow analyses, and a demonstration shows that
it can be used for modular analysis.
We also empirically evaluate the running time improved by staging.
The experiment shows an order of magnitude speedup with staging for
control-flow analyses.

\iffalse
We obtain a sound static analysis, specialized for
a given program, that runs faster. More surprisingly, we show that by applying
the staged abstract interpreter to \textit{open} programs and considering the
free variables as dynamic inputs, we obtain a modular analysis that generates
sound partial analysis results which can be composed and reused later without
losing precision, even though the original abstract interpreter is a
whole-program analysis algorithm.

Based on the idea of staged abstract interpreters, we show several case studies,
including \citeauthor{Boucher:1996:ACN:647473.727587}'s abstract compilation of
0-CFA, pushdown control-flow analysis with context-sensitivity and precise
stores, and a numerical analysis on an imperative language.

We empirically evaluate the performance improvements on control-flow analysis of
benchmark programs. The results show speedups up to 2.3x with staging on a
monovariant analysis.
\fi

% It is well known that a staged interpreter is a compiler, which provides
% performance improvement by specializing the interpreter to a given program. In
% this paper, we study \textit{abstract} interpreters combined with multi-stage
% programming, i.e., the staged abstract interpreters. By staging the abstract
% interpreter with respect to a program, we obtain a specialized analysis that
% runs faster. By applying the staged abstract interpreter with \textit{open}
% programs and considering the free variables as dynamic inputs, we obtain a
% modular analysis that generates sound partial analysis results which can be
% composed and reused later without losing precision, though the original
% abstract interpreter is a whole-program analysis algorithm. Using the idea of
% staged abstract interpreters, we show several case studies, including
% \citeauthor{Boucher:1996:ACN:647473.727587}'s abstract compilation of 0-CFA,
% pushdown control-flow analysis with context/path/flow-sensitivity and
% store-widening, and a numerical analysis on an imperative language. We also
% empirically evaluate the improvement of performance on control-flow analysis
% of benchmark programs. The result shows an average speedup of Nx when staging
% to Scala for a monovariant analysis, and Mx for polyvariant analysis.

\end{abstract}


%% 2012 ACM Computing Classification System (CSS) concepts
%% Generate at 'http://dl.acm.org/ccs/ccs.cfm'.
\begin{CCSXML}
<ccs2012>
<concept>
<concept_id>10011007.10011006.10011008</concept_id>
<concept_desc>Software and its engineering~General programming languages</concept_desc>
<concept_significance>500</concept_significance>
</concept>
<concept>
<concept_id>10003456.10003457.10003521.10003525</concept_id>
<concept_desc>Social and professional topics~History of programming languages</concept_desc>
<concept_significance>300</concept_significance>
</concept>
</ccs2012>
\end{CCSXML}

\ccsdesc[500]{Software and its engineering~General programming languages}
\ccsdesc[300]{Social and professional topics~History of programming languages}
%% End of generated code


%% Keywords
%% comma separated list
% \keywords{keyword1, keyword2, keyword3}  %% \keywords are mandatory in final camera-ready submission


%% \maketitle
%% Note: \maketitle command must come after title commands, author
%% commands, abstract environment, Computing Classification System
%% environment and commands, and keywords command.
\maketitle

\iffalse

\renewcommand\thefootnotecopyrightpermission{}
\footnotetextcopyrightpermission{
Preprint, November 2018.\\ Copyright held by the authors.}
\fancyhead[RO,LE]{Preprint, November 2018}

\fi

\section{Introduction} \label{intro}

Statically analyzing semantic properties of a program is a
widely-known undecidable problem. Abstract interpretation as a
lattice-based approach to sound static analyses was proposed
by \citet{DBLP:conf/popl/CousotC77}. Equipped with Galois connections,
the analyzer can obtain the program runtime behaviors approximately
by computing the fixed points on the abstract domain. Despite the tremendous
theoretical development of abstract interpretation over the years,
constructing artifacts and analyzers that perform sound abstract
interpretation for modern and expressive languages was considered
abstruse and complicated for a long time.

In recent years, there are rich progress on the methodologies for
constructing abstract interpreters from systematic principles, instead
of ad-hoc engineering. A notable one is the Abstracting Abstract
Machine (AAM) methodology \cite{DBLP:journals/jfp/HornM12,
  DBLP:conf/icfp/HornM10}. It uncovers an approach to derive sound
abstract interpreters from their concrete counterparts, for example,
abstract machines, where the soundness can be easily established by examining
the transformation of semantic artifacts. For example, the CEK machine
\cite{DBLP:conf/popl/FelleisenF87} for concrete execution can be
readily refactored to an effective $0$-CFA control-flow analysis
\cite{Shivers:1988:CFA:53990.54007, Midtgaard:2012:CAF:2187671.2187672}:
first we tweak the environment dereference as a nondeterministic choice,
such that the environment may contain mutiple possible values for a varaiable,
then allocating continuations in the environment to exploit return-flow information,
and also constrain the addresses space to be finite.
This systematic abstraction approach can be tailored to
different language features (such as states, first-class controls,
exceptions and concurrency) and sensitivity analyses
\cite{DBLP:conf/icfp/Gilray0M16, DBLP:conf/popl/GilrayL0MH16,
  Darais:2015:GTM:2814270.2814308}. It also has been applied to various
small-step abstract machines \cite{DBLP:journals/jfp/HornM12,
  DBLP:conf/icfp/HornM10, Sergey:2013:MAI:2491956.2491979} and
big-step definitional interpreters \cite{Wei:2018:RAA:3243631.3236800,
  DBLP:journals/pacmpl/DaraisLNH17, Keidel:2018:CSP:3243631.3236767}.

Based on the idea of abstracting abstract machines, more pragmatically,
several implementation strategies utilizing purely functional programming
to build abstract interpreters were emerged. Such techniques include monads,
monad transformers \cite{DBLP:journals/pacmpl/DaraisLNH17, Sergey:2013:MAI:2491956.2491979},
arrows \cite{Keidel:2018:CSP:3243631.3236767}, extensible effects \cite{Githubsemantic} and etc.
The pure approaches provide certain benefits. The abstract interpretation
artifacts can be built compositionally and modularly, e.g., by using monad
transformers. Therefore, the soundness of analysis can be proved with less efforts,
either by mechanized \cite{Darais:2016:CGC:2951913.2951934} or paper-based proof
\cite{Keidel:2018:CSP:3243631.3236767}. Also, referential transparent and
equational reasoning allow programmers to reason the correctness of their
implementations more confidently.

However, besides the intrinsic complexity of static analysis, there are
additional abstraction penalties with these high-level implementation approaches.
First, similar to concrete interpreters, the abstract interpreter analyzes the
program by traversing the abstract syntax tree, which poses an interpretive
overhead, such as the pattern matching on the ASTs and recursive calls on the sub
expressions. Those kind of overhead can be negligible if the abstract
interpreter only runs on the program for one time, but also can be accumulated
significantly if it runs repeatedly, for example, on libraries.
Second, the abstract interpreter written in pure languages usually extensively
uses effect systems to implement the semantics of abstract interpretation.
For example, the abstract interpreter that returns all possible runtime values
is a form of nondeterminism, where nondeterminism monads can help.
Although such pure approaches have its own merits and elegance,
compared with imperative stateful implementations, they are significantly slower.

In this paper, we propose an abstraction-without-regret approach to
eliminate those performance penalties for abstract interpreters, meanwhile
still keeping the benefits come from purely functional programming.
In short, our approach applies ideas and techniques from program specialization
and embedded DSLs to abstract interpreters. 1) Using multi-stage
programming , we can specialize the abstract interpreter with respect to an
input program and then generate efficient low-level code that does the actual analysis.
The result of specialization is reusable, and the effect layers have been
eliminated in the generated code. 2) Inspired by the tagless-final interpreters,
we define a generic interpreter that abstracts over binding-time and
different semantics, which allows user to implement different semantics
modularly, including the staged abstract interpretation semantics.
Together with the multi-stage programming and type-level annotations of binding-times,
we can derive the staged abstract interpreters without intrusive changes to
the corresponding unstaged one, thus the soundness is untouched.
In this sense, our approach is no regret of both performance and engineering
effort. We elaborate these two main ideas in detail as follow.

\paragraph{Futamura Projection of Abstract Interpreters}

The idea of specializing interpreters can be traced to Futamura
projections \cite{Futamura1999, futamura1971partial}.
The first Futamura projection specifically shows that
specializing an interpreter w.r.t. the input program yields an
equivalent executable. For instance, @eval@ is an interpreter for
some language:
\begin{lstlisting}
  def eval(e: Expr)(arg: Input): Value
\end{lstlisting}
Given a program $e_0$ of type @Expr@, by applying the specialization,
we can obtain a specialized interpreter $\texttt{eval}_{\texttt{e0}}$ :
@Input@ $\to$ @Value@, which is the so-called \textit{equivalent executable}.
By definition of the interpreter, they produce the same result when applied
with the argument $\texttt{eval}_{\texttt{e0}}(arg) = [\![ e_0 ]\!] arg $.

Partial evaluation \cite{DBLP:books/daglib/0072559} was the first
proposed approach to realize Futamura projections: it first discovers
the binding-times (\textit{static} or \textit{dynamic}) of variables,
then evaluates the static part of the program, and generates a residualized
program that solely relies on the dynamic part. However, it is hard to
precisely analyze binding-times given an arbitrary program.
As an alternative approach to specialization, multi-stage programming (MSP)
\cite{taha1999multi, DBLP:conf/pepm/TahaS97}
lets the programmer to explicitly control and annotate the
binding-times in the program, then the MSP system will check whether these annotations
are consistent and specialize the program using that information.
The staging annotations can be either syntactic (e.g., quote and quasiquote
in MetaML/MetaOCaml) or type-based (e.g., the Lightweight Modular Staging
framework \cite{DBLP:conf/gpce/RompfO10} in Scala).

Our proposed framework adopts the type-based multi-stage programming from
the Lightweight Modular Staging framework and implements the Futamura
projection of a big-step abstract interpreter for a small stateful,
higher-order language.
We use \citet{DBLP:journals/pacmpl/DaraisLNH17}'s abstract definitional
interpreter that uses monad transformers as the unstaged baseline. By
deriving staged monads that can be used to do code generation, we may
obtain a staged abstract interpreter that fulfills the specialization.
After staging, the generated code is specialized to the input program,
and all monadic operations are all inlined and compiled down to low-level
Scala code.

\input{fig_confluence.tex}

\paragraph{Generic Interpreter and Reinterpretation}

Program specialization and abstract interpretation are two orthogonal concepts.
To implement the confluence of them, we first construct a generic interpreter that
is agnostic to both binding-times and value domains used in the semantics.
Later, the generic interpreter can be instantiated from these two dimensions (Figure~\ref{confluence}):
\begin{itemize}
\item With a flat binding-time and concrete domains, it is a ordinary definitional interpreter based
  on big-step operational semantics;
\item With two-level binding-times and concrete domain, it is a compiler that translate to program
  intro antoher language;
\item With a flat binding-time and abstract domains, it is an definitional abstract interpreter
  \cite{DBLP:journals/pacmpl/DaraisLNH17} that statically computes runtime properties;
\item With two-level binding-times and abstract domains, it is an optimizing program analyzer but
  works in the fashion of compilation.
\end{itemize}

Although the four artifacts may look like very differently at first glance,
but in fact are all firmly rooted in the concrete semantics of the language.
This observation provides a way to abstract over the interpreter and achieve
the flexibility of reinterpreting the shared interpreter. We adapt the monadic
interpreter schema, which already have been applied to abstract interpreters,
and introduce the binding-time abstraction into the generic interpreter.
The generic interpreter returns a value of monadic type, which can be varied
by different semantics. The domain of the interpreter and the effects such as read,
state and nondeterminism can be both wrapped into this monadic type.
The binding-time abstraction is represented by a higher-kinded type,
and controls whether the interpreter produces values directly or generates code.
It is worth to mention that the binding-time type is also instrumented into
the monadic type, so that we will distinguish normal monads and staged monads.

\paragraph{Applications and Evaluations}
We evaluate the idea of staging an abstract interpreter through
case studies and an empirical evaluation on performance.
1) We compare our approach with abstract compilation
\cite{Boucher:1996:ACN:647473.727587}, a implementation technique for
control-flow analyses, and show that by utilizing type-based stage
annotations we can achieve the same optimization, and meanwhile,
the analyzer does not need to change, thereby requires much less
engineering efforts.
2) We extend the basic staged abstract interpreter to different flow
analyses, including a store-widened analysis, a context-sensitive
analysis and abstract garbage collection.
3) We notice that staging an abstract interpreter enables modularly
compiling an analysis to programs. Here we borrow the concept of modular
analysis, and show that the compiled analysis is reusable.
Therefore the approach provides a modular way to create optimized analysis
code by mechanized reusing a whole-program analyzer.
4) We also empirically evaluate the performance improved by staging,
showing an order of magnitude speed-up on flow-analysis \todo{explicit note the number}.

\paragraph{Contributions} Briefly, the contribution of the paper is as follows:
\begin{itemize}[leftmargin=2em]
  \item To our best knowledge, we present the first one to use
    multi-stage programming to specialize a general abstract interpreter,
    and our approach does not touch the soundness of analyses.
  \item Intellectually, our framework naturally extends the first
    Futamura projection to abstract interpreters, showing a
    well-grounded approach to optimize static analyses via
    metaprogramming.
  \item Practically, we show that staging an abstract interpreter is
    useful to improve performance and scalability of analyses by case
    studies and an empirical evaluation.
\end{itemize}

\paragraph{Organization} The paper is organized as follows:
\begin{itemize}[leftmargin=2em]
  \item We begin by introducing our target language and reviewing
    monads in Scala, and then presenting the generic interpreter (Section~\ref{prelim}).
    After which, we review its instantiations of concrete interpretation
    (Section~\ref{unstaged_conc}) and staged concrete interpretation
    (Section~\ref{stagedinterp}).
  \item We present the unstaged abstract interpreter under the same framework by
    replacing the environment, store and values to their abstract counterparts (Section~\ref{unstaged_abs}).
    After, we show the combination of approximation and specialization, dubbed
    \textit{staged abstract interpreters}, can be readily derived (Section~\ref{sai}).
  \item We conduct three case studies (Section~\ref{cases_study}), which demonstrate that
    our approach requires less engineering efforts, is applicable to various analysis,
    and enables compiling analysis modularly.
  \item We empirically evaluate the performance improvement by staging on
    control-flow analysis (Section~\ref{evaluation}). We compare the both
    context-insensitive and store-widening analysis.
\end{itemize}

\iffalse
On the other side, static analysis is a tradeoff between performance and
precision: higher precision usually leads to longer running time.

4. Existing method to improve the performance is adhoc, engineering heavy, require to rewrite the optimized version, therefore harder to reason about the correctness
6. program analyzers are also meta-programs, they manipulate other programs as data objects
\fi

\newcommand{\TLang}{$L_\lambda$}

\section{Preliminaries} \label{prelim}

In this section, we first describe the abstract syntax of the language for our interpreter, 
then present the generic interpreter shared among the four different
semantics, after which, we instantiate the interpreter to the concrete one.
It is worth noting that we choose to use Scala and monad to demonstrate the
idea, but the approach is not restricted to our choice. One can use
imperative or direct style in other MSP languages (e.g., MetaOCaml
\cite{DBLP:conf/gpce/CalcagnoTHL03, DBLP:conf/flops/Kiselyov14} and Template
Haskell \cite{Sheard:2002:TMH:636517.636528} to construct such staged abstract
interpreters.

\subsection{Abstract Syntax} \label{bg_lang}

We consider a call-by-value $\lambda$-calculus in direct-style, extended
with numbers, arithmetic, recursions, and conditionals. Other effectful features
such as assignments can also be supported readily.
%In Section~\ref{cases_imp}, we will add more imperative features to the language.
Since we are mostly interested in analyzing the dynamic behaviors of the
program, we disguise any static semantics and type system. We also assume that
input programs are well-typed and all variables are distinct. The abstract
syntax is shown as follows:

\begin{lstlisting}
  abstract class Expr
  case class Lit(i: Int) extends Expr                         // numbers
  case class Var(x: String) extends Expr                      // variables
  case class Lam(x: String, e: Expr) extends Expr             // abstractions
  case class App(e1: Expr, e2: Expr) extends Expr             // applications
  case class If0(e1: Expr, e2: Expr, e3: Expr) extends Expr   // conditionals
  case class Rec(x: String, rhs: Expr, e: Expr) extends Expr  // recursions
  case class Aop(op: String, e1: Expr, e2: Expr) extends Expr // arithmetic
\end{lstlisting}

The abstract syntax we present in fact can be seen as a deep embedding of the
language -- we use data-types to represent programs. This design choice allows us
to easily use different interpretations over the AST; with the inheritance and
overriding mechanism in Scala, we may also add new language constructs and reuse
existing interpretations \todo{cite Bruno?}.

\iffalse
We will give the concrete semantics using a big-step definitional
interpreter. The interpreter is a recursive function that takes the program AST,
environment, and store, and returns the evaluated value and the accompanying
store. The environment is a mapping from identifiers to addresses, and the store
is a mapping from addresses to values. We use the store to model recursion and
mutation in concrete semantics; it is also useful for polyvariant analysis. This
environment-and-store-passing style big-step interpreter is standard and can
also be obtained by refunctionalizing \cite{DBLP:conf/ppdp/AgerBDM03,
Wei:2018:RAA:3243631.3236800} a small-step CESK machine
\cite{DBLP:conf/popl/FelleisenF87}.
\fi

\subsection{Monads in Scala} \label{monadscala}

A monad is a type constructor @M[_]: * -> *@ with two operations, often called
@return@ and @bind@. Informally, @return@ wraps a value into the monad @M@, and
@bind@ unwraps the monadic value and transforms it into a new monadic value.
Pragmatically in Scala, we define a monad type class using trait @Monad@ (Figure
\ref{fig:monad}), where it declares the @pure@ \footnote{We elect to use
\texttt{pure} as the name, since \texttt{return} is a keyword in Scala and
\texttt{unit} is a built-in function in LMS.} and @flatMap@ operation. The trait
itself takes the monad type @M[_]@ as argument, which is a higher-kinded type
that takes a type and returns a type. The method @pure@ promotes values of type
@A@ to values of type @M[A]@. The monadic @bind@ operation is usually called
@flatMap@ in Scala, which takes a monad-encapsulated value of type @M[A]@, a
function of @A => M[B]@ and returns values of type @M[B]@.

\begin{figure}[h!]
  \centering
  \begin{subfigure}[b]{0.55\textwidth}
    \begin{lstlisting}
  trait Monad[M[_]] {                                  
    def pure[A](a: A): M[A]                            
    def flatMap[A,B](ma: M[A])(f: A => M[B]): M[B]     
  }                                                    
    \end{lstlisting}
    \caption{trait \texttt{Monad}} \label{fig:monad}
  \end{subfigure}
  ~
  \begin{subfigure}[b]{0.4\textwidth}
    \begin{lstlisting}
trait MonadOps[M[_], A] {
  def map[B](f: A => B): M[B]
  def flatMap[B](f: A => M[B]): M[B]
}
    \end{lstlisting}
    \caption{trait \texttt{MonadOps}} \label{fig:monadops}
  \end{subfigure}
\end{figure}

Similar to Haskell's @do@-notation, Scala provides special syntactic support for
monadic operations through @for@-comprehension.
For example, an object of @List[A]@ is an instance of @List@ monad, where @A@ is the element type. 
Then to compute the Cartesian product of two lists of numbers, we can use Scala's
@for@-comprehension syntax.

\begin{lstlisting}
  val xs = List(1, 2); val ys = List(4, 5)
  for { x <- xs; y <- ys } yield (x, y) // List((1,4), (1,5), (2,4), (2,5))
\end{lstlisting}

The Scala compiler will translate the above @for@-comprehension expression into
an equivalent one using @flatMap@ and @map@ ~\cite{scala_spec}. The last binding
in the @for@-comprehension is translated into a @map@, where the expression of
@yield@ becomes the body expression of that @map@ application. The foregoing
bindings in the comprehension are all translated into calls of @flatMap@.

\begin{lstlisting}
  xs.flatMap { case x => ys.map { case y => (x, y) } }
\end{lstlisting}

Note that here the monadic object @List[_]@ encapsulates the data internally.
Therefore it only exposes the simplified version of @flatMap@, where the
monad @M[A]@ is not introduced as a function argument. The trait @MonadOps@
(Figure \ref{fig:monadops}) defines the simplified version of monadic
operations that are necessary for @for@-comprehension. 
The conversion between @Monad@ and @MonadOps@ can be done by using the implicit design pattern.
In the rest of the paper, we use Scala's @for@-comprehension syntax and monad
transformers such as @ReaderT@, @StateT@, and @ListT@ to write our interpreters.
The implementation of monads and monad transformers essentially borrows the
ground-truth from Haskell.

\subsection{Generic Interpreter} \label{generic_if}

Monad transformers are type constructors of kind @(* -> *) -> (* -> *)@, which
take a monad as argument and produces another monad. By using monad
transformers, we can combine multiple monads into a single one. Constructing
extensible interpreters using monad transformers was first proposed by
\citet{DBLP:conf/popl/LiangHJ95}, and later has been applied to abstract
interpreters \cite{Sergey:2013:MAI:2491956.2491979,
DBLP:journals/pacmpl/DaraisLNH17, Darais:2015:GTM:2814270.2814308}. In this section,
with the multi-stage programming and monad transformers in mind while leaving them
as abstract type members, we present the generic interface of a big-step
definitional interpreter.

\paragraph{Basic Types} We start with some basic type definitions used in the
interpreter. The identifiers in the program are represented by strings. The two
default components in the interpreter are environments @Env@ and stores @Store@,
i.e., mappings from identifiers to addresses and mappings from addresses to
values, respectively. The @Env@ models accessible variables, and @Store@ models
the persistent heap through the program runtime. But the @Addr@ and @Value@ are
just declared as abstract types.

\begin{lstlisting}
  trait Semantics {
    type Ident = String; type Addr; type Value
    type Env = Map[Ident, Addr]; type Store = Map[Addr, Value]
    type R[_] // Binding-time as a higher-kinded type
    ... // The definitions in the rest of this section are enclosed in trait Semantics.
  }
\end{lstlisting}

\paragraph{Binding-time Abstraction} As mentioned before, the binding-time is
declared as a higher-kinded type @R[_]@. If we simply instantiate @R@ as an identity
type (i.e., @type R[T] = T@), then the generic interpreter will execute the program.
In Section \ref{stagedinterp}, we will instantiate @R@ using LMS's built-in
next-stage type annotation @Rep@, which makes the interpreter act as a compiler.

\paragraph{Monadic Operations} We define the return type of the interpreter as
@Ans@, which is an arbitrary monad type @AnsM[_]@ wrapping the type @Value@. 
As mentioned in Section ~\ref{monadscala}, to use the @for@-comprehension
syntax, certain constraints have to be added on the type @AnsM@. Here, we use a
structural type @MonadOps@ to require @AnsM@ to at least implement @map@ and
@flatMap@. It is worth noting that @MonadOps@ takes another type parameter
@R[_]@ as binding-time; accordingly the outer @R@ defined in the trait is passed
to @MonadOps@. Inside of @MonadOps@, @R[_]@ wraps the data types @A@ and @B@ that are
encapsulated by the monad, but not the monad type @M@ itself. When acting as
compilers, we will also replace the monads to the ones that work on staged
values.

We also define several methods to obtain and modify the environment and store.
These methods return monadic values of type @AnsM[_]@, which encapsulate the 
environment or store, or simply a @Unit@ value for effects.
For example, @local_env@ takes a value @ans@ of type @Ans@ and an environment 
which will be installed when evaluating @ans@.

%\vspace{-1em}
\begin{figure}[h!]
  \centering
  \begin{subfigure}[b]{0.45\textwidth}
    \begin{lstlisting}
  type MonadOps[R[_], M[_], A] = {
    def map[B](f: R[A] => R[B]): M[B]
    def flatMap[B](f: R[A] => M[B]): M[B]
  }
  
  type AnsM[T] <: MonadOps[R, AnsM, T]
  type Ans = AnsM[Value]
    \end{lstlisting}
  \end{subfigure}
  ~
  \begin{subfigure}[b]{0.55\textwidth}
    \begin{lstlisting}
// Environment operations
def ask_env: AnsM[Env]
def local_env(ans: Ans)(ρ: R[Env]): Ans
// Store operations
def get_store: AnsM[Store]
def put_store(σ: R[Store]): AnsM[Unit]
def set_store(av: (R[Addr], R[Value])): AnsM[Unit]
    \end{lstlisting}
  \end{subfigure}
\end{figure}
%\vspace{-1em}

\paragraph{Primitive Operations} Next we define several primitive operations.
Two versions of @alloc@ are declared. The first one takes a store and an
identifier and produces a fresh address of non-monadic type @R[Addr]@. Since
the freshness of the address may depend on the store, which might be a
next-stage value as indicated by its type, the type of addresses is also wrapped
by @R[_]@. The other one would simply wrap the address with our monadic type
@AnsM[_]@.
\begin{lstlisting}
  def alloc(σ: R[Store], x: Ident): R[Addr];  def alloc(x: Ident): AnsM[Addr]
\end{lstlisting}

Other primitive operations provide basic functionality for the interpreter.
The method @num@ and @close@ deal with primitive values, which lift literal
terms (e.g., lambdas) to our value representation (e.g., closures).
The method @get@ simply retrieves the value mapped from the identifier @x@ in
the environment and store. Conditionals and arithmetic is handled by @br0@
and @arith@, respectively. The methods @ap_clo@ takes a function value and an
argument value and then does the application. Note that the @Env@, @Store@, and
@Value@ are all wrapped by @R[_]@ since they will be known as next-stage value
when acting as compilers.

%TODO: close/ap_clo takes an ev

\begin{lstlisting}
  def num(i: Int): Ans
  def get(σ: R[Store], ρ: R[Env], x: Ident): R[Value]
  def close(ev: Expr => Ans)(λ: Lam, ρ: R[Env]): R[Value]
  def br0(test: R[Value], thn: => Ans, els: => Ans): Ans
  def arith(op: Symbol, v1: R[Value], v2: R[Value]): R[Value]
  def ap_clo(ev: Expr => Ans)(rator: R[Value], rand: R[Value]): Ans
\end{lstlisting}

\paragraph{The Interpreter} Now we can define the semantics-agnostic interpreter
in monadic form, shown in Figure \ref{fig:shared_int}.
The essential idea is to traverse the abstract syntax tree while maintaining the
effects such as reader and state.
It is worth noting that the interpreter is written in open-recursive style -- it
can not refer to itself directly, instead, @eval@ takes an additional parameter
@ev@ of type @Expr => Ans@ to refer to itself. Accordingly, the method @close@
that lifts lambda terms to closures and @ap_clo@ that applies functions also
takes an extra @ev@.

%\vspace{-1em}
\begin{figure}[h!]
  \centering
  \begin{lstlisting}
          def eval(ev: Expr => Ans)(e: Expr): Ans = e match {
            case Lit(i) => num(i)                   case Let(x, rhs, e) => for {
            case Var(x) => for {                      v  <- ev(rhs)
              ρ <- ask_env                            ρ  <- ask_env
              σ <- get_store                          α  <- alloc(x)
            } yield get(σ, ρ, x)                      _  <- set_store(α → v)
            case Lam(x, e) => for {                   rt <- local_env(ev(e))(ρ + (x → α))
              ρ <- ask_env                          } yield rt
            } yield close(ev)(Lam(x, e), ρ)         case Aop(op, e1, e2) => for {
            case App(e1, e2) => for {                 v1 <- ev(e1)                                               
              v1 <- ev(e1)                            v2 <- ev(e2)
              v2 <- ev(e2)                          } yield arith(op, v1, v2)
              rt <- ap_clo(ev)(v1, v2)              case Rec(x, rhs, e) => for {
            } yield rt                                α  <- alloc(x)
            case If0(e1, e2, e3) => for {             ρ  <- ask_env
              cnd <- ev(e1)                           v  <- local_env(ev(rhs))(ρ + (x → α))
              rt  <- br0(cnd, ev(e2), ev(e3))         _  <- set_store(α → v)
            } yield rt                                rt <- local_env(ev(e))(ρ + (x → α))
                                                    } yield rt                    
          }
  \end{lstlisting}
\caption{The generic interpreter,
  shared by the unstaged/staged + concrete/abstract interpreter.}
\label{fig:shared_int}
\end{figure}
%\vspace{-1em}

Since the interpreter is written in an open-recursive style, we declare an abstract
combinator @fix@ used to close the recursion. For concrete-interpretation instantiation, it
works like the Y combinator; for abstract-interpretation instantiation, we will
instrument the interpreter by defining a memorized version of @fix@ to ensure
termination. Finally, a top-level wrapper @run@ is declared; the return type
@Result@ depends on what kind of monad we will be using, this is also an abstract type.

\begin{lstlisting}
  def fix(ev: (Expr => Ans) => (Expr => Ans)): Expr => Ans
  type Result; def run(e: Expr): Result
\end{lstlisting}

%==========================================================================

\section{A Concrete Interpreter} \label{unstaged_conc}

Now we can instantiate the interpreter as a standard environment-store
interpreter. Such interpreter also can be obtained by refunctionalization of
CESK machines \cite{Felleisen:1987:CAH:41625.41654, DBLP:conf/ppdp/AgerBDM03}.

\paragraph{Concrete Components}
The two types we need to concretize are @Addr@ and @Value@, afterward, the @Env@
and @Store@ is concretized automatically. To assure the freshness, we use @Int@
for the address space @Addr@. A value can be either a tagged number
\texttt{IntV}, or a closure \texttt{CloV} that contains a lambda term and an
environment. The final return value of the interpreter is a pair of values and
stores, where the @Value@ and @Store@ are obviously next-stage objects. We also
provide a standard fixed-point combinator to close the open-recursive function @ev@.

\begin{lstlisting}
  trait ConcreteComponents extends Semantics {
    type Addr = Int
    sealed trait Value
    case class IntV(i: Int) extends Value
    case class CloV(λ: Lam, e: Env) extends Value
    type Result = (R[Value], R[Store])
    def fix(ev: (Expr => Ans) => (Expr => Ans)): Expr => Ans = e => ev(fix(ev))(e)
  }
\end{lstlisting}
%\vspace{-1em}

\paragraph{Unstaged Monads}
The environment and store can be modeled by reader effect and state effect,
i.e., a Reader monad and a State monad. We combine them using monad transformers
@ReaderT@ and @StateT@.
In other words, we instantiate the @AnsM@ monad as a stack of @ReaderT@ and @StateT@
transformers\footnote{The question mark syntax is a kind projector
  \cite{kindprojector}, thus \texttt{StateT[IdM,Store,?]} is equivalent to \newline
  \texttt{(\{type M[T]=StateT[IdM,Store,T]\})\#M}}, where the @ReaderT@ is
parameterized by type @Env@, @StateT@ is parameterized by type @Store@, and the
inner-most monad @IdM@ is just an identity monad.

\begin{lstlisting}
  trait ConcreteSemantics extends ConcreteComponents {
    type R[T] = T
    type AnsM[T] = ReaderT[StateT[IdM, Store, ?], Env, T]
    ...
  }
\end{lstlisting}

Here we sketch the basic idea of unstaged @ReaderT@ and @StateT@. Readers may
refer to \cite{DBLP:conf/popl/LiangHJ95, Chiusano:2014:FPS:2688794} for more detail 
and implementation of monad transformers.
A @ReaderT@ monad transformer encapsulates the computation @R => M[A]@, where
@R@ is the reader type, @M[_]@ is a monad type. The @ReaderT@ monad returns the
transformed value of type @M[A]@. Similarly, a @StateT@ monad encapsulates the
computation @S => M[(A, S)]@, where @S@ is the state type, @M[_]@ is a monad
type. The @StateT@ monad returns the transformed value of type @M[(A, S)]@,
where the new state is accompanied.
Note that the binding-time type @R@ is an identity type, thus the monads are
operated on unstaged data. We can also see this from the signature of @flatMap@:
the function @f@ takes an unstaged value of type @A@ and produces a monadic
value.

\begin{lstlisting}
  case class ReaderT[M[_]: Monad, R, A](run: R => M[A]) {
    def flatMap[B](f: A => ReaderT[M, R, B]): ReaderT[M, R, B] = ... }
  case class StateT[M[_]: Monad, S, A](run: S => M[(A, S)]) {
    def flatMap[B](f: A => StateT[M, S, B]): StateT[M, S, B] = ... }
\end{lstlisting}

Then, the monadic operations that manipulate the environment and store are
simply constructing the proper monads and lifting it to the top-level of our
monad stack. To modify the store, for example, we may have a @StateT@ monad that
transforms the current store $\sigma$ to @σ + αv@, which results in
@StateT[IdM, Store, Unit]@, and then lift this @StateT@ value to @ReaderT@,
i.e., @AnsM[Unit]@.
\begin{lstlisting}
  def set_store(αv: (Addr, Value)): AnsM[Unit] = liftM(StateTMonad.mod(σ => σ + αv))
\end{lstlisting}

\paragraph{Primitive Operations}
Other primitive operations can be implemented straightforwardly, for example,
@alloc@ simply takes the size of the current store and produces the successor number.
We elide most of them but describe a little bit on how do we handle functions
and applications as they will be very different when staging and abstract
interpretation is involved. The method @close@ takes a lambda term and an
environment, then create a closure value @CloV@.

\begin{lstlisting}
  def close(ev: Expr => Ans)(λ: Lam, ρ: Env): Value = CloV(λ, ρ)
\end{lstlisting}

To apply a function, the method @ap_clo@ takes a function value and an argument
value, and extracts the lambda term and environment enclosed in the function value,
and evaluates the body expression of that lambda term under the new environment and store.

\begin{lstlisting}
  def ap_clo(ev: Expr => Ans)(rator: Value, rand: Value): Ans = rator match {
    case CloV(Lam(x, e), ρ: Env) => for {
      α <- alloc(x)
      _ <- set_store(α → rand)
      rt <- local_env(ev(e))(ρ + (x → α))
    } yield rt
  }
\end{lstlisting}

With other primitive operations implemented, we can implement the top-level
@run@ method, where $\rho_0$ and $\sigma_0$ are the initial empty environment and
store.

\begin{lstlisting}
  def run(e: Expr): Result = fix(eval)(e)(ρ$_0$)(σ$_0$)
\end{lstlisting}

\input{30_stageconc}
\input{40_absinterp}
\section{From Abstract Interpreters to Staged Abstract Interpreters} \label{sai}

In the previous sections, we have seen an unstaged abstract interpreter and a
staged concrete interpreter, now we begin describing the implementation of their
confluence -- a staged abstract interpreter.
Unsurprisingly, the staged abstract interpreter in this section has the same
abstract semantics with the unstaged version from Section~\ref{unstaged_abs}.
The approach from unstaged to staged abstract interpreter is modular,
principled, and does not sacrifice soundness or precision. The designer of the
analyzer has no need to rewrite the analysis. We first present the staged
lattices and staged monads, and then the staged version of primitive operations,
especially @close@, @ap_clo@ and @fix@, at last, we discuss several optimizations
and our design.

\subsection{Staged Lattices}

In Section \ref{stagedpoly_lat}, we exploited the higher-kinded type @R@ to
achieve stage polymorphism. Now we instantiate the type @R@ to @Rep@ and
still use power sets as an example to present its staged version.

\begin{lstlisting}
  trait RepLattice[S] extends Lattice[S, Rep]
  def RepSetLattice[T]: RepLattice[Set[T]] = new RepLattice[Set[T]] {
    lazy val ⊥: Rep[Set[T]] = Set[T]()
    lazy val ⊤: Set[T] = error("No representation for ⊤")
    def ⊑(l1: Rep[Set[T]], l2: Rep[Set[T]]): Rep[Boolean] = l1 subsetOf  l2
    def ⊔(l1: Rep[Set[T]], l2: Rep[Set[T]]): Rep[Set[T]]  = l1 union     l2
    def ⊓(l1: Rep[Set[T]], l2: Rep[Set[T]]): Rep[Set[T]]  = l1 intersect l2
  }
\end{lstlisting}

As we can see, the @RepSetLattice@ is an instance of @RepLattice@, which is also
an abstraction of the underlying data structure. The lattice operations
eventually are delegated to @subsetOf@, @union@ and @intersect@ on
@Rep[Set[T]]@. We even don't have to change the implementation code, but just
the types – from @Set[T]@ to @Rep[Set[T]]@. LMS library provides an implicit
conversion lifting to next-stage values. In the code generation part, these
operations emit their corresponding next-stage code. Again, other lattice
structures such as products and maps are implemented in a similar way.

\subsection{Staged Abstract Semantics}

Now we implement the staged abstract semantics, where @R@ is instantiated as
@Rep@, and the abstract components are shared from the unstaged version.

\paragraph{Staged Monads for Abstract Interpretation}
We use the same monad stack structure as in the unstaged abstract interpreter,
but replacing \textit{all} the transformers to their staged version. The
following code shows this change. For the readability, we squeeze the three
inner transformers into a single monad @RepSetReaderStateM[A, B, C]@, where @A@
and @B@ are both @Cache@ when used in @AnsM@. Now the result type is a pair of
two staged value: @Rep[Set[(Value, Store)]]@ and @Rep[Cache]@.

\begin{lstlisting}
  trait StagedAbstractSemantics extends AbstractComponents {
    type R[T] = Rep[T]
    type RepSetReaderStateM[A, B, C] = RepSetT[RepReaderT[RepStateT[RepIdM, A, ?], B, ?], C]
    type AnsM[T] = RepReaderT[RepStateT[RepSetReaderStateM[Cache, Cache, ?], Store, ?], Env, T]
    type Result = (Rep[Set[(Value, Store)]], Rep[Cache])
    ...
  }
\end{lstlisting}

Since our monad stack is uniformly using staged data, so the staged @SetT@ now
stores a staged value of type @Rep[Set[A]]@, inside of another staged monad @M@.

\begin{lstlisting}
  case class SetT[M[_]: RepMonad, A](run: M[Set[A]]) {
    def flatMap[B: Manifest](f: Rep[A] => SetT[M, B]): SetT[M, B] = ...
  }
\end{lstlisting}

\paragraph{Primitive Operations} We keep our road map that focuses on functions
and applications. The @close@ method now is a rough mixing of the staged
concrete and unstaged abstract version: we have a current-stage function @f@,
that takes four next-stage values, @in@ and @out@ additionally; inside of @f@,
we collapse the @Ans@ monad to values of type @Result@ by providing the desired
arguments, i.e., the new environment, new store, and caches. The collapse
happens at current-stage, so the call of @ev@ is unfolded after staging. At
last, we generate a singleton set containing the compiled closure
@emit_compiled_clo(f)@, represented by an IR node in LMS.

\begin{lstlisting}
  def emit_compiled_clo(f: (Rep[Value], Rep[Store], Rep[Cache], Rep[Cache])
                           => Rep[(Set[(Value,Store)], Cache)]): Rep[AbsValue]
  def close(ev: Expr => Ans)(λ: Lam, ρ: Rep[Env]): Rep[Value] = {
    val Lam(x, e) = λ
    val f: (Rep[Value],Rep[Store],Rep[Cache],Rep[Cache]) => Rep[(Set[(Value,Store)],Cache)] = {
      case (arg, σ, in, out) =>
        val α = alloc(σ, x); val ρ_* = ρ + (unit(x) → α); val σ_* = σ ⊔ Map(α → arg)
        ev(e)(ρ_*)(σ_*)(in)(out)
    }; Set[AbsValue](emit_compiled_clo(f))
  }
\end{lstlisting}

The @ap_clo@ method is also similar: we use the staged version of @lift_nd@ to
lift the set of closures into the monad stack, and generate a next-stage value
using @emit_ap_clo@, which represents the result of application from future.
Finally, we put the future @out@ cache, store, and values back into the
current-stage monads. The @emit_ap_clo@ method takes the target closure @clo@,
argument @rand@, store, and caches.

\begin{lstlisting}
  def lift_nd[T](vs: Rep[Set[T]]): AnsM[T]
  def emit_ap_clo(rator: Rep[AbsValue], rand: Rep[Value], σ: Rep[Store],
                  in: Rep[Cache], out: Rep[Cache]): Rep[(Set[(Value, Store)], Cache)]
  def ap_clo(ev: Expr => Ans)(rator: Rep[Value], rand: Rep[Value]): Ans = for {
    σ <- get_store;     clo <- lift_nd[AbsValue](rator)
    in <- ask_in_cache; out <- get_out_cache
    val res: Rep[(Set[(Value, Store)], Cache)] = emit_ap_clo(clo, rand, σ, in, out)
    _ <- put_out_cache(res._2)
    vs <- lift_nd[(Value, Store)](res._1)
    _ <- put_store(vs._2)
  } yield vs._1
\end{lstlisting}

\paragraph{Staged Caching and Fixpoint Iteration} 

Our fixed-point iteration again relies on the two caches @in@ and @out@, which are
both staged values now. Therefore, the query of whether the @out@ cache contains
the current configuration produces a next-stage Boolean value, i.e.,
@Rep[Boolean]@. Consequently, the branching cannot be determined statically --
we need to generate code for @if@. Figure \ref{fig:staged_coind_cache} shows the
staged version of @fix@.

\begin{figure}[h!]
  \centering
\begin{lstlisting}
  def fix(ev: (Expr => Ans) => (Expr => Ans)): Expr => Ans = e => for {
    ρ <- ask_env; σ <- get_store; in <- ask_in_cache; out <- get_out_cache
    val cfg: Rep[Config] = (unit(e), ρ, σ)
    val res: Rep[(Set[(Value, Store)], Cache)] =
      if (out.contains(cfg)) (out(cfg), out) // a next-stage if expression
      else { val m: Ans = for {
               _ <- put_out_cache(out + (cfg → in.getOrElse(cfg, ⊥)))
               v <- ev(fix(ev))(e)
               σ <- get_store
               _ <- update_out_cache(cfg, (v, σ))
             } yield v
             m(ρ)(σ)(in)(out) }
    _ <- put_out_cache(res._2); vs <- lift_nd(res._1); _ <- put_store(vs._2)
  } yield vs._1
\end{lstlisting}
\caption{The staged co-inductive caching algorithm.}
\label{fig:staged_coind_cache}
\end{figure}

The @res@ variable is a next-stage result, which consists of a next-stage @if@
expression. The true branch simply returns a pair of queried result form the
@out@ cache and the @out@ cache. The else branch constructs a monad @m@ first,
which evaluates @e@ under the new @out@ cache. After which, we collapse the
monad @m@ with desired arguments. Finally, we obtain a placeholder @res@ since
it is a next-stage value, and put the content of @res@ back into the monad
stack.

\paragraph{Code Generation} The IR nodes for compiled closures and
applications are defined as follows:

\begin{lstlisting}
  case class IRCompiledClo(f: (Rep[Value], Rep[Store], Rep[Cache], Rep[Cache])
                           => Rep[(Set[(Value, Store)], Cache)], λ: Lam, ρ: Rep[Env]) extends Def[AbsValue]
  case class IRApClo(clo: Rep[AbsValue], arg: Rep[Value], σ: Rep[Store],
                     in: Rep[Cache], out: Rep[Cache]) extends Def[(Set[(Value, Store)], Cache)]
\end{lstlisting}

The code generator for these two IR nodes is similar to staged concrete
interpreter, which generates a next-stage @CompiledClo@ object and a next-stage
function call to @f@, respectively.
                   
\subsection{Optimizations} \label{staged_ds}

\iffalse
Revision: Solving Practical Challenges.
theoretically, all the things should work nicely. Unfolding the interpreter over the AST.
But the generated code is blowed up. For example .... This should not affect the correctness, 
but poses burden on the MSP system (ie LMS) and the next stage compiler/runtime (ie, scalac and JVM).
1) LMS becomes slower since a large IR graph is contructed during the staging.
2) Scalac becomes slower when reading a such large source code.
JVM has certain limitation on the size of a single method.

TODO: can we formulate selective caching as a partially-static data law.
TODO: lambda lifting for if
\fi

Our staging schema works well in theory, but in practice would suffer from code
explosion, and possibly runtime GC overhead (at the next stage) when analyzing
(specializing) large programs. In this section, we present several optimizations
that largely mitigate these issues. Still, implementing these optimizations do
not need to change the generic interpreter.

\paragraph{Specialized Data Structures}

Although every component except @Expr@s are staged, we treat the data
structures such as @Map@s as black-boxes, which means the operations on a @Map@
directly become code in the next stage, but we don't inspect any further inside.
As we identified when introducing the generic interface, the keys of @Env@s are
identifiers (i.e., strings) in the program, which are completely known
statically. This leaves us a chance to further specialize the data structures.
For example, let's assume that the @Map[K, V]@ is implemented as a hash map. If
the keys of type @K@ are all known statically, then the indices also can be
computed statically. Thus the specialized map would be an array of type
@Array[Rep[V]]@, whose elements are next-stage values; the size of the array is
also known statically. An access to the map is translated into an access to the
array with pre-determined index during staging.

Particularly, if we are specializing a monovariant analysis, the address space
is equivalent to the set of identifiers in the program. Then the accesses to the
environment can be completely determined statically and generates the addresses
directly. After which, the stores can be specialized as arrays of @Rep[Value]@
elements.

\paragraph{Selective Caching} It is observed that the two-fold co-inductive
caching is used for every recursive call of our abstract interpreter. But this
is not necessary and even very expensive when generating code for atomic
expressions such as literals or variables, as they always terminate. Borrowing
the idea from the partition of A-Normal Form \cite{Flanagan:1993:ECC:155090.155113},
we may use a selective caching algorithm:

\begin{lstlisting}
  def fix_select: Expr => Ans = e => e match {
    case Lit(_) | Var(_) | Lam(_, _) => eval(fix_select)(e)
    case _ => fix_cache(e)
  }
\end{lstlisting}

\paragraph{Partially-static Data}

Our treatment to binding-times is coarse-grained: @Expr@s is static, the rest of
the world are all dynamic. But this is not always true since the static data
has to be used somewhere with the dynamic operation. Partial-static data is a
way to improve binding-times and optimize the generated code.

For example, to fold a singleton set (often appears in @SetT@), e.g.,
@Set(x).foldLeft(init)(f)@ where @x@ and @init@ are staged values, a naive code
generator would faithfully apply @foldLeft@ to the set with an anonymous
function. But we can also utilize the algebraic property of @foldLeft@ to
generate cheaper code: @Set(x).foldLeft(init)(f) = f(init, x)@. Since the
function @f@ is known at the current stage, we completely eliminate the fold
operation and function application. We apply several rewritings enabled by
partial-static data structures, such as @Set@ and @Map@, which greatly reduces
the size of residual programs.

%\paragraph{Heterogeneous Staging} The generated program is in A-Normal form. \todo{TODO}

\section{Discussion}

We have gradually presented the confluence of specialization and abstraction of
concrete interpreters. In this section, we review and summarize our recipe
to achieve the staged abstract interpreter, discuss the correctness issue
and different design choices.

\subsection{Summarizing the Approach}

\todo{TODO}

\paragraph{What has been eliminated?} In the generated code, all the
primitive operations (such as @eval@, @fix@, @ap_clo@, etc.) and monadic
operations (such as @flatMap@ and @map@) are eliminated. The residual program
consists of statements and expressions that purely manipulate the environment,
store, and two caches, whose underlying representations are all @Rep[Map[K,V]]@. We
also have several operations on tuples and lists, which are residualized from
the use of monads internally.

\todo{may be show an small example?}

\subsection{Correctness}

\todo{TODO}

As one of the merits of our approach, the staging does not compromise any
soundness, because we only change the underlying monads and data structures to
the staged version, not any part of the abstract semantics. Based on the
assumption that MSP system and staged data structure preserve the equivalence
during staging, we are confident that the staged abstract interpreter does the
same analysis as the unstaged one.

\subsection{Design Choices}

\paragraph{Are Monads Necessary?} No. One can always inline the monads and
obtain an abstract interpreter in continuation-passing style, or simply use
explicit side-effects such mutation to implement the artifact with the same
abstract semantics. In either case, we can still apply the staging schema to
the abstract interpreter. As evidence, we provide a staged abstract
interpreter written in direct-style in the accompanying artifact.

\paragraph{Big-step vs Small-step}

What we implemented is a big-step, compositional abstract interpreter in monadic
style, where \textit{compositional} means that every recursive call of our abstract
interpreter is applied to proper substructures of the current syntactic
parameters \cite{10.1007/3-540-61580-6_11}. This compositionality ensures that
specialization can be done by unfolding, as well as guarantees the termination
of specialization procedure. It is also possible to specialize small-step
operational abstract semantics through abstract compilation
\cite{Boucher:1996:ACN:647473.727587} -- as
\citet{Johnson:2013:OAA:2500365.2500604} implemented it for
optimizing Abstract Abstract Machines. However, the generated abstract
byte-code still requires another small-step abstract machine to execute, which is
an additional engineering effort.

\input{60_cases}
\section{Empirical Evaluation} \label{evaluation}

To evaluate how the performance can be improved by our staged abstract interpreters
approach, we scale our toy interpreter to a large subset of the Scheme language
and evaluate the analyzing time against with the unstaged versions. The result
shows that the staged versions are averagely 10+ times as fast as the unstaged
version, as strong evidence of our \textit{abstraction without regret}
approach.

\paragraph{Implementation and Evaluation Environment}
We implement the abstract interpreters for control-flow analysis, i.e., a
value-flow analysis that computes which lambda terms can be called at each
call-sites in functional languages. With other suitable abstract domains, the
abstract interpreter also can be extended to other analyses. A front-end
desugars Scheme to a small core language that makes the analyzer easy to
implement. As described in Section \ref{cfa}, we implement two monovariant
0-CFA-like analysis, one is equipped with store-widening, the other does not.
\todo{why store widening?} Our prototype implementation currently generates
Scala code; the generated Scala code will be compiled and executed also on JVM.
it is possible to also generate C/C++ code in the future. In the experiments, we
implement several optimizations mentioned in Section \ref{staged_ds},
specifically, the selective caching and rewriting rules exploiting
partially-static data. These optimizations are useful to reduce the size of
generated code.

We use Scala 2.12.8 and Oracle JVM 1.8.0-191 \footnote{The options for running
  JVM is set to the following: \texttt{-Xms2G -Xmx8G -Xss1024M
  -XX:MaxMetaspaceSize=2G - XX:ReservedCodeCacheSize=2048M}}, running on an Ubuntu 16.04 LTS
(kernel 4.4.0) machine. All of our evaluations were performed on a machine with 4 Intel
Xeon Platinum 8168 CPU at 2.7GHz and 3 TiB of RAM. Although the machine has 96
cores (192 threads) in total, the abstract interpreters are single-threaded.
To minimize the effect of warming-up from the HotSpot JIT compiler, all the
experiments are executed for 20 times and we report the statistical median value
of the running times. We set a 5 minutes timeout.

\paragraph{Benchmarks}
The benchmark programs we used in the experiment are collected from several
previous papers \cite{Johnson:2013:OAA:2500365.2500604, ashley:practical,
DBLP:journals/corr/abs-1102-3676} and existing artifacts
\footnote{https://github.com/ilyasergey/reachability} for control-flow analysis.
Some of the benchmarks are small programs designed for experiments, such as the
@kcfa-worst-@$n$ series, which are intended to challenge $k$-CFA; while some
other benchmarks are from real-world applications, for examples, the RSA public
key encryption algorithm @rsa@. In the Figure \ref{evaluation_result}, we report
the number of AST node after parsing and desugaring (excluding comments) as a
proper measurement of program size.

\paragraph{Result}

\begin{figure}[h]
\footnotesize
\begin{tabular}{@{}ll|lll|lll@{}}
\toprule
    program             &\#AST & unstaged   & staged     & $\frac{\text{unstaged}}{\text{staged}}$ & unstaged   & staged    & $\frac{\text{unstaged}}{\text{staged}}$  \\ 
    \midrule
                        &      & \multicolumn{3}{c}{w/o store-widening}  &  \multicolumn{3}{c}{w/ store-widening}\\
    \midrule
    @fib@               & 32   & 3.288 ms   & 0.154 ms   & 21.33x      & 1.434 ms   & 0.098 ms  &  14.62x       \\
    @rsa@               & 451  & 238.171 ms & 23.333 ms  & 10.20x      & 11.977 ms  & 1.197 ms  &  10.00x       \\
    @church@            & 120  & 61.001 s   & 4.277 s    & 14.26x      & 2.338 ms   & 0.534 ms  &  4.37x        \\
    @fermat@            & 310  & 23.540 ms  & 2.885 ms   & 8.05x       & 7.146 ms   & 0.915 ms  &  7.81x        \\
    @mbrotZ@            & 331  & 665.456 ms & 66.070 ms  & 10.07x      & 11.008 ms  & 1.476 ms  &  7.45x        \\
    @lattice@           & 609  & 29.230 s   & 2.627 s    & 11.12x      & 16.432 ms  & 2.427 ms  &  6.76x        \\
    @kcfa-worst-16@     & 182  & 44.431 ms  & 3.211 ms   & 13.83x      & 4.425 ms   & 0.850 ms  &  5.20x        \\
    @kcfa-worst-32@     & 358  & 284.268 ms & 9.065 ms   & 31.35x      & 10.109 ms  & 1.661 ms  &  6.08x        \\
    @kcfa-worst-64@     & 710  & 2.057 s    & 0.029 s    & 70.23x      & 23.269 ms  & 3.312 ms  &  7.02x        \\
    @solovay-strassen@  & 523  & 5.078 s    & 0.766 s    & 6.62x       & 18.757 ms  & 3.142 ms  &  5.96x       \\
    @regex@             & 550  & -          & -          & -           & 6.803 ms   & 1.088 ms  &  6.24x       \\
    @matrix@            & 1732 & -          & -          & -           & 85.611 ms  & 9.297 ms  &  9.20x       \\
    \bottomrule
\end{tabular}
\caption{Evaluation result for monovariant control-flow analysis.} \label{evaluation_result}
\end{figure}

Figure~\ref{evaluation_result} shows the evaluation result, comparing the
performance improved by staging on two monovariant CFA (without and with store-widening). The
\textit{unstaged} and \textit{staged} columns show the median time to finish the
analysis. The column $\frac{\text{unstaged}}{\text{staged}}$
shows the improvement from staging. The dash - in some cells indicates timeout.
As we can see from the table, the staged version significantly outperforms the
unstaged version on all benchmarks.
On average, the staging produces code that runs 17x times faster
than the unstaged analyzer.

We also observe that much overhead of unstaged analyzers are from the monadic
layers. Thus our evaluation result provides an abstraction without regret
approach to construct optimizing abstract interpreters -- the user may write a
high-level modular and composable abstract interpreter, and then derive a staged
version in a principled way that runs significantly faster.

\todo{analyze some interesting examples}
\section{Related Work}

\paragraph{Optimizing Static Analysis Through Specialization}
The underlying idea in this paper is closely related to abstract compilation (AC)
\cite{Boucher:1996:ACN:647473.727587}: removing the interpretation overhead on
traversing the syntax tree by specialization. Specifically, the residual program
of AC can be either textual or closures. However, as we studied in Section
\ref{cs_ac}, with the perspective from generative programming and monadic
interpreters, we make it more general, extensible and easy.
\citet{Johnson:2013:OAA:2500365.2500604} adapt the idea of closure
generation to optimize small-step abstract interpreter in state-transition
style. The analyzed program is firstly compiled to an IR called "abstract
bytecode", which are actually higher-order functions and will be executed later
on an abstract abstract machine for that IR. \citet{damian1999partial} provides a
formal treatment to abstract compilation and Shiver's CFA, as well as proofs to
establish the correctness of the certified specialized analyzer.
\citet{amtoft1999partial} applied partial evaluation for constraint-based
control flow analysis. Splitting an analysis into multiple stages is also
studied for analysis other than for control-flow, though the formulation may
very different. For example, \citet{DBLP:conf/cgo/HardekopfL11} apply staging to
flow-sensitive pointer analysis. The first stage is to analyze the program code
to obtain a sparse representation, and then the second stage conducts the
flow-sensitive analysis based on the first one.
The abstract domains can be also specialized w.r.t. the static program
structure, e.g., decomposing polyhedras \cite{DBLP:conf/popl/SinghPV17,
Singh:2017:PCD:3177123.3158143} to smaller ones depending on the variables
involved such that the abstract transformers can compute using less time and
space.

\paragraph{Abstract Interpreters} Abstract interpretation was proposed as a
semantic-based approach to build sound static analysis by approximation
\cite{DBLP:conf/popl/CousotC77}. As to build semantic artifacts, the Abstracting
Abstract Machines (AAM) \cite{DBLP:journals/jfp/HornM12, DBLP:conf/icfp/HornM10}
approach shows that abstract interpreters can be derived systematically from concrete
semantic artifacts. The AAM approach is closely related to control-flow analysis
for higher-order languages \cite{Midtgaard:2012:CAF:2187671.2187672}.
Using monads to construct abstract interpreters is explored by
\citet{Sergey:2013:MAI:2491956.2491979} and
\citet{DBLP:journals/pacmpl/DaraisLNH17, Darais:2015:GTM:2814270.2814308}.
The unstaged abstract interpreter in this paper follows the abstracting definitional
interpreters approach from \citet{DBLP:journals/pacmpl/DaraisLNH17}.
Similar to definitional abstract interpreters, \citet{Wei:2018:RAA:3243631.3236800}
reconstruct big-step abstract interpreters with delimited continuations;
\citet{Keidel:2018:CSP:3243631.3236767} present a modular arrow-based abstract
interpreter that makes proof of soundness can be constructed compositionally.
\citet{DBLP:conf/cc/CousotC02} proposed an abstract interpretation framework for
modular analysis. We borrow the notation of modular analysis, \todo{but intentional properties
  are different, compare with summary}

\paragraph{Two-level Semantics} The idea of reinterpreting the semantics as
abstract interpretation can be traced to Nielson's two-level semantics
\cite{NIELSON1989117}; using two-level semantics for code generation was also
explored by \citet{NIELSON198859}. \citet{Sergey:2013:MAI:2491956.2491979}'s
work of monadic abstract interpreters is also closely related to the two-level
semantics: the use of a generic interface with monads and then reinterpret it by
different semantics is already two-level. Instead of focusing on semantics, this
paper shows how a staged analyzer can be built and used to increase efficieny of
static analysis, we augment the monadic abstract interpreter from another
dimension, using code generation to produce efficient low-level code.

\paragraph{Partial Evaluation and Multi-stage Programming}
Partial evaluation as an automatic technique for program specialization
was studied comprehensively by \citet{10.1007/3-540-61580-6_11, DBLP:books/daglib/0072559}.
In this paper, we use mutli-stage programming as an approach to
program specialization. The Lightweight Modular Staging framework \cite{DBLP:conf/gpce/RompfO10}
we used in the paper relies on type-level stage annotations.
Other notable implementations of MSP exist in ML family, e.g.,
MetaML \cite{DBLP:conf/pepm/TahaS97} and MetaOCaml
\cite{DBLP:conf/gpce/CalcagnoTHL03, DBLP:conf/flops/Kiselyov14}.
The idea of staging an abstract interpreter we presented in this paper is also
applicable to other MSP systems.
Multi-stage programming has been widely used to improve the performance in many
domains, such as optimizing compilers or domain-specific languages
\cite{DBLP:conf/pldi/RompfSBLCO14, DBLP:conf/snapl/RompfBLSJAOSKDK15,
DBLP:journals/tecs/SujeethBLRCOO14, DBLP:conf/gpce/SujeethGBLROO13,
DBLP:journals/jfp/CaretteKS09}, numerical computation \cite{PGL-038,
DBLP:conf/pepm/AktemurKKS13}, generic programming
\cite{DBLP:journals/pacmpl/Yallop17, Ofenbeck:2017:SGP:3136040.3136060}, data
processing \cite{DBLP:conf/oopsla/JonnalageddaCSRO14,
DBLP:conf/popl/KiselyovBPS17}, query compilation in databases
\cite{DBLP:conf/osdi/EssertelTDBOR18, DBLP:conf/sigmod/TahboubER18}, etc.

As an source of inspiration of this paper, Futamura projections reveals a
hierarchy and close relation between interpreters and compilers, which was
originally proposed by \citeauthor{futamura1971partial} in 1970s\cite{futamura1971partial},
and later reprinted \cite{Futamura1999}.
\citeauthor{Amin:2017:CTI:3177123.3158140} consider a tower of concrete
interpreters and how to collapse them by using MSP -- it would be
interesting to explore this idea for multiple layers of abstract interpreters
\cite{Cousot:2019:AAI:3302515.3290355, Giacobazzi:2015:APA:2676726.2676987}.
\citet{10.1007/11561347_18} discussed combining
multi-stage programming with functors and monads. But they didn't \todo{monad transfomers?}
Similar to the idea in this paper, specializing monadic interpreters was
explored by \citet{DBLP:conf/dsl/SheardBP99, danvy1991compiling}. \todo{what's the difference?}

\paragraph{Control-flow Analysis} useful applications, why important

\input{90_conc}

%% Acknowledgments
% Ack Fei, Greg, James, Qianchuan, Guanhong
\begin{acks}                            %% acks environment is optional
                                        %% contents suppressed with 'anonymous'
  %% Commands \grantsponsor{<sponsorID>}{<name>}{<url>} and
  %% \grantnum[<url>]{<sponsorID>}{<number>} should be used to
  %% acknowledge financial support and will be used by metadata
  %% extraction tools.
  This material is based upon work supported by the
  \grantsponsor{GS100000001}{National Science
    Foundation}{http://dx.doi.org/10.13039/100000001} under Grant
  No.~\grantnum{GS100000001}{nnnnnnn} and Grant
  No.~\grantnum{GS100000001}{mmmmmmm}.  Any opinions, findings, and
  conclusions or recommendations expressed in this material are those
  of the author and do not necessarily reflect the views of the
  National Science Foundation.
\end{acks}


%% Bibliography
\bibliography{references}

% %% Appendix
% \appendix
% \section{Appendix}

% Text of appendix \ldots

\end{document}
